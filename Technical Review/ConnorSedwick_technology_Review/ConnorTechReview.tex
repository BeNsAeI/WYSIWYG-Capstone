\documentclass[journal,10pt,onecolumn,compsoc]{IEEEtran} \usepackage[margin=1.0in]{geometry} \usepackage{pdfpages} \usepackage{graphicx} 
\usepackage{listings}
\usepackage{verbatim}
\graphicspath{/graphics} \setlength{\parskip}{\baselineskip} \setlength\parindent{24pt}
\usepackage[english]{babel}
%\usepackage{fullpage}

\title{VisualFlow Tech Review}
\author{Group 33: Connor Sedwick}
\date{\today}

\begin{document}
\maketitle
\begin{abstract}
The purpose of this document is to compare and contrast technologies that may be implemented in the development of our graphical user-interface system. 
The technologies reviewed range from basic programming languages to third-party APIs developed to aid in GUI design and development.
Technologies are reviewed and chosen on how well they fit with the design of our graphical user-interface system.
\end{abstract}
\newpage
\tableofcontents
\newpage
\section{Introduction}
My name is Connor Sedwick. I am responsible for the variable handling and abstraction pieces of our software. 
I also have the task of researching and managing the design of the menu interface for our GUI software.
I also manage the documentation of the VisualFlow project.

\section{Variable handling}
Our software is meant to handle a lot of values. 
It must also handle many different types of values. 
For our software to run, it needs to be based off of a programming language that makes it easy to implement variables and supports numerous types. 
As it stands, there are many programming languages and each implements variables in different ways. 
This section will discuss the Python, C/C++, and Java programming languages and our decision on what language to use for our software and the reasoning for it.

%\subsection{C/C++}
\noindent C and C++ offer very traditional syntax when it comes to variables. 
The C language is also supported on nearly every device. 
With the requirement of our software being portable there is a definite plus to using this programming language. 
That being said, let us look at the syntax for it.

\begin{lstlisting}
	// C/C++ supports multiple variable types

	// To create a variable in C, you must define its type like so:

	int integerVariable;
	integerVariable = 4;

	char characterVariable;
	characterVariable = 'c';
    
	float floatingPointVariable;
	floatingPointVariable = 4.5;

	double veryLargeNumber;
    	veryLargeNumber = 10000000;

	String stringVariable; //This one is only supported in later versions of C.
	stringVariable = "Hello World";

	//Example of passing variable to function
    
	int functionEx(int temp, char temp2, float temp3);
	functionEx(integerVaraible, characterVariable, floatingPointVariable);
\end{lstlisting}

\noindent C/C++ supports a number of variable types and has very simple syntax for defining variables. 
C/C++ is what is called a strong typed language.
This means that the variables must be defined and given a data type in order to be used. \cite{cppvar} 
Functions in C/C++ must also be defined with what data types they are taking in.
%When considering that our GUI must be able to interpret user values entered and translate them into syntax this could be problematic when using C/C++. 
%\subsection{Java}
Like C/C++, Java is also supported on a host of devices. 
Unlike C/C++, Java comes packaged with string support.
Interestingly enough, Java and C/C++ share much of the same syntax.

\begin{lstlisting}
	// Java is similar to C/C++ when it comes to defining variables

	// To create a variable in Java, you must define its type like so:

	int integerVariable = 11;

	char characterVariable = 'c';

	//To use floats in Java, they must be type-cast. 
	float floatingPointVariable = (float)4.5; 
    
	//Double also acts as a floating point number.
	double veryLargeNumber = 3.4;

	String stringVariable = "Hello world";
    
	//Java also supports boolean values
	bool qualifier = true;
	bool disqualifier = false;
    
	class Fun_example{
    	public int exampleFunc(int n);
	}
\end{lstlisting}

\noindent Whereas older versions of C require the user to define and then provide values to variables, Java can do it all in one line. \cite{javavar} 
C does have the upper-hand in this situation when it comes to defining floating point variables as it does not require the use of type-casting which could prove troublesome when implemented and translated from a GUI.
Our GUI is also supposed to support user-defined objects or classes.
Java requires that functions be defined within classes.\cite{javafunction}
This feature in a sense forces users to create classes for organizational purposes of their functions.
Considering that our software is meant to support object-oriented designs this could prove to be a very useful choice over C.

%\subsection{Python}
\noindent Compared to C and Java, Python is a very different creature.
Python, being a dynamic programming language offers much flexibility for the user.
To put it in perspective an example is provided below for the syntax of variable defining in Python.

\begin{lstlisting}
	integerVariable = 11

	characterVariable = 'c'

	floatingPointVariable = 4.5

	stringVariable = "Hello world"

	qualifier = True;

	disqualifier = False;
    
	#function definition
	def printinfo( name, age ):
		"This prints a passed info into this function"
		print "Name: ", name
		print "Age ", age
		return;
	
	#call printinfo function
	printinfo( age=50, name="miki" )
\end{lstlisting}

\noindent Take note that there is no required defining of variable type before setting the variable to a value. \cite{pythonvar} 
This makes variables in Python very easy to implement.
Considering that our GUI software will have users creating variable blocks for the flowchart to use in their functions it seems much better to have an overarching "variable" block rather than needing to create separate blocks for each type of variable as would be the case with C or Java.
Function parameters in Python are much simpler than Java and C/C++ for this reason as well. 
Instead of setting hard parameters that must be met by the value being passed to a function, Python just requires that a value is passed. \cite{pythonfunction} 
For these reasons, we should use Python as our primary language for variable handling.

\newpage
\section{Abstraction}

Given that the GUI that we are creating is meant to provide an abstracted view of a user's software to them, the language we use should mirror our GUI closely.
Python, C/C++, and Java all support object oriented programming. 
Java does especially due to it requiring every function be defined within a class.
All languages require constructors in order for classes to be invoked. 
The main difference, as usual, is the syntax:

\begin{lstlisting}
	#Python
	>>> class Dog:
	def __init__(self, name, type):
		self.name = name
		self.type = type

	>>> x = Dog('Millie', 'Boston Terrier')
	>>> x.name 
	'Millie'
	>>> x.type
	'Boston Terrier'
----------------------------------------------------------
	//Java
	public class Bicycle{
    	
        int gear;
        int cadence;
        int speed;
        
		public Bicycle() {
			gear = 1;
			cadence = 10;
			speed = 0;
		}
        
		public static void showParts(){
			System.out.println("Gear = " + gear);
			System.out.println("Cadence = " + cadence);
			System.out.println("Speed = " + speed);
		}
	}

	Bicycle mybike = new Bicycle();
    mybike.showParts();
----------------------------------------------------------
	//C++
	class Line {
		public:
			void setLength( double len );
			double getLength( void );
			Line();  // This is the constructor
 
		private:
			double length;
		};
    
	Line::Line(void){
	length = 20;
	cout << "Object is being created" << endl;
	}
    
    Line::getlength( void ){
	return length;
    }
    
    void main(){
	Line myLine;
	cout << "Length is " << myLine.getlength() << endl;
    }
\end{lstlisting}

\noindent Constructors in Python are simple in that they are defined as "\_\_init\_\_()". 
C++ by contrast requires that constructors be defined with the same name as the class. \cite{cclass} 
Java requires that the user call the constructor when they are initializing an object whereas in C++ it only need be defined as that object type as the constructor is implicitly called each time a variable of that type be created. \cite{javaclass} 

\noindent The simpler the implementation of objects in our software the better. 
As such, Python has the upper hand. 
In C++, for example, unless a variable is set as public within a class it cannot be displayed unless a function is called that returns it's value. \cite{cclass} 
This provides assurance that values cannot be changed unless deliberately through function calls that do so.
Python allows users to reference values just by referring to the "self." variables set in the constructor. \cite{pythonclass} 
C++ classes could require that we implement more features to our GUI to accommodate public and private values as well as special formatting for public versus private functions.

\noindent C++ and Java are very similar in their syntax. 
Both require functions to access and display class values. 
Both also support public and private variable types. \cite{cclass} \cite{javaclass} 
By contrast, Python does not use private variables. 
This simplifies accessing values and effectively decreases the amount of code required to access each individual variable. 
If a user were to create a class in our GUI using the C++ implementation it would require that they create a function for every single private variable of that class. 
Depending on the object they are trying to model that could range between 1 to 100 or so access functions just to reach variable values.
This compounded with the pieces representing variables alone would easily overpopulate the interface.

\noindent Keeping in mind that functions defined for classes in C++ must also include scope prefixes this would require that we have a means of storing the class name as a prefix to append to class functions. 
Python instead requires all functions used by a class be defined and written under the class header. \cite{pythonclass} 
Here it merely comes down to a styling choice. 
C++ does a better job of organizing the parts that make up each class into sections containing public and private function definitions and variables.
Python makes each class a giant listing of functions and variables.
The other important aspect to consider is whether our software can support a Deep Learning API. 
Considering Google's TensorFlow API is written to support the Python programming language it's obvious that Python would be best suited for our purpose.

\newpage
\section{GUI Representation}

\noindent For our software to work in the manner we want it to we must choose a framework that supports the design choices we have decided on.
The framework we choose must support drag and drop functionality and multiple pages (preferably a tabbing function).
Considering that Python is best suited for our software we should look to using a framework for our GUI that is supported by Python.
Python has many frameworks specifically developed with it in mind.
A few main frameworks that are cross-platform are libavg, Kivy, and wxPython.
All three choices listed are supported by Windows, MacOS and Linux systems. 
All three APIs also support buttons whether they appear as images or text.

\noindent Both Kivy and libavg support touch and drag functions due to them primarily being designed for mobile application and game development. \cite{Kivy}\cite{libavg} 
Libavg, specifically, supports multitouch.
While our software is meant as a desktop application, having the ability to use it on a tablet would also increase its portability.
If individuals working on a project wanted to be able to collaborate and build a solution together, supporting multitouch would also be an interesting feature to consider.
wxPython supports drag and drop as well. 
wxPython's documentation also provides examples of the implementation of a drag and drop feature. \cite{wxPython} 

\noindent Due to our software supporting a build space function, there is a need to ensure that the framework we use can track coordinates of blocks.
libavg itself has what is called a hierarchy of coordinate systems. \cite{libavg} 
This means that users can create divs or cut out portions of the interface that have their own relative coordinate system.
This could prove useful for our build space design as it can allow us to set a relative portion of the interface to be used exclusively for node placement.
Kivy supports relative coordinates as well and the website for it provides much more detail on the matter. \cite{Kivy} 
wxPython does not have a specific feature built in for dealing with relative coordinates. \cite{wxPython}
Instead, it requires that the developer do some arithmetic in order to ensure values fall into the correct coordinate plane.

\noindent In order to make efficient use of one of the mentioned frameworks we must have some form of documentation to aid us in using it.
wxPython itself has multiple books written on it. %Find citations and insert here
wxPython's API manual page is host to extensive documentation.
The website for Kivy also hosts a great deal of well-organized API documentation.
libavg, does provide documentation on its website, but is very sparse and unorganized.
Provided the timeframe we have to work on this project it would be beneficial to use an API that is well documented. 
Its manual should also be easy to refer to (i.e. have a search feature).

\noindent The less work we need to do to build our GUI and get it running the better.
With this in mind, using wxPython would be very time consuming as it would require much time spent developing algorithms for coordinate tracking.
libavg and Kivy would appear better choices as they have support for relative coordinate systems built into them.
However, when it comes to documentation, libavg comes up short when compared to Kivy and wxPython.
Taking these factors into consideration it would appear most beneficial to use the Kivy API for our GUI development.

\newpage
\bibliographystyle{IEEEtran}
\bibliography{sources}
\end{document}