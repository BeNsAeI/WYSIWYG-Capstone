\documentclass[journal,10pt,onecolumn,compsoc]{IEEEtran}
\setcounter{page}{101}
\usepackage[margin=1.0in]{geometry} 
\usepackage{pdfpages} 
\graphicspath{/graphics} 
\setlength{\parskip}{\baselineskip} \setlength\parindent{24pt}
\usepackage[english]{babel}
%\usepackage{fullpage}
\title{Learning Outcomes from the Project}
\author{Behnam Saeedi}
\begin{document}
\maketitle
\section{What technical information did you learn}
\noindent A great knowledge of a wide variety of technical skills was acquired during the course of this project. These skill include, time management, concurrent development, module testing, programing techniques and debugging. The team was responsible of analyzing the problem, coming up with solutions, breaking the solutions to concurrent development sets, implement, test and debug.

\section{What non-technical information did you learn}
\noindent Other than technical skills, a great deal of interpersonal communication was required. Team members needed to take responsibility for critical tasks and step up in order to prevent catastrophic failure for the team. Despite the intense workload, some team members pushed themselves harder in order to achieve their short and long term goals. The most important discovery was that the process to solving a problem is:
\begin{enumerate}
\item Identifying the question
\item Identifying the capabilities of our tools
\item Learning how to look up features of our tool-set
\item Look at examples
\item Implement our own solution
\item Test
\item Debug
\end{enumerate}
\noindent This is a very scalable method to approach our problems. It can work from high level solution for the entire problem. It can also apply in each individual step in building our software.

\section{What have you learned about project work?}
\noindent Project work is series smaller problem solving steps that accumulates over the course of time into a set of solutions that together tackle an aspect of the specific problem.

\section{What have you learned about project management}
\noindent Timing, Communication and concurrency had the biggest impact in management of our project. In fact most difficult problems to solve that we faced were related to one of these elements of project management. Project management needs to be consistent and worked out from the very beginning.

\section{what have you learned about working in teams}
\noindent Working in team is difficult. A team's effectiveness heavily depend on work load organization and task management. Some level of hierarchy is required in order for the team to function properly and being proactive is extremely important. A team-member that is constantly waiting for others to tell them what to do decreases the performance of the team. Also, pick reasonable goals in documentation. However always try to achieve a better goal and hope by time the goal is achieved, the quality is comparable with the original set goal.

\section{if you could do it all over, what would you do differently?}
\noindent There are several things that I would assert if I had to redo the capstone project:
\begin{itemize}
\item \textbf{Mandatory meetings:} Set a weekly meeting time for all members to meet
\item \textbf{Weekly Client meeting and emails:} I would meet with the client every week going through what is done and what is not, setting ARs for next week's meeting. (Weekly checkpoints).
\item \textbf{Spend more time researching:} Our project suffered from lack of documentation for one of our APIs. It made it extremely difficult for us to et anything done. I would very much like to avoid that.
\item \textbf{Identify each member's skills:} It is important to know who is better at doing a specific task in the team. This will greatly increase efficiency in the project procedure.
\item \textbf{Communication:} Talk to teammates, TAs, Instructors and client. Many problems get solved by simply introducing a new perspective .
\end{itemize}
\end{document}
